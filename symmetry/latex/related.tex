\section{Related work}
\label{sec:related}

Symmetry detection and analysis have been thoroughly studied from theoretical, algorithmic, and applications aspects for many years. Papers on this topic can  be classified according to whether they consider \emph{global} or \emph{partial} symmetries; most but not all recent papers consider approximate rather than exact symmetries.
We aim to detect partial approximate symmetry in 3D shapes in this paper, and so we concentrate on the most closely related 3D symmetry detection methods, as well as ideas relating segmentation and symmetry. Comprehensive reviews can be found in~\cite{xu2009,berner2011}.

A common approach to symmetry detection is to identify the global symmetry in Euclidean space from clusters in the transformed space~\cite{podolak2006,pauly2008}.
Other approaches~\cite{Berner2008,bokeloh2009} formulate symmetry detection as graph matching, based on  reliable line features extracted from scanned data.
Extrinsic symmetry methods employ invariance under rigid transformations and possibly scaling for symmetry detection~\cite{zabrodsky1997,podolak2006,ovsjanikov2008,ovsjanikov2010,chertok2010,hooda2011},
and for finding patterns~\cite{pauly2008,bokeloh2009,yeh2009}.
Ovsjanikov et al. perform global intrinsic symmetry detection for non-rigid isometric shapes  
using eigenanalysis of the surface Laplace-Beltrami operator~\cite{ovsjanikov2008},
or by use of heat kernel maps~\cite{ovsjanikov2010}.
Two robust symmetry detection algorithms proposed by Kim et al~\cite{kim2010,kim2011} utilize the Mobius transformation and conformal mapping.
Most algorithms determine discrete symmetries by sampling. An exception is the work~\cite{ben-chen2010}
where global intrinsic symmetry is detected using the approximate Killing vector field~\cite{ben-chen2010} from a continuous perspective.

Recent works have focused on analyzing partial and approximate symmetries, which is more difficult than finding global symmetries. As for global symmetry, partial symmetry detection can also be categorized as extrinsic~\cite{mitra2006,bronstein2009,berner2011} or intrinsic~\cite{lasowski2009,xu2009,mitra2010,raviv2010,bronstein2011}.
%%%RRM You have not really explained what intrinsic and extrinsic mean
%%%ZQC add some exlain in previous paragraph
A global multidimensional scaling (GMDS) approach devised by Bronstein and coworkers~\cite{bronstein2006,bronstein2009,raviv2010,bronstein2011} can detect  extrinsic and intrinsic symmetries; it has been mainly applied to  shape  retrieval applications based on invariants. Other researchers have proposed  methods based on a variety of techniques, e.g.\ Markov random field models~\cite{lasowski2009}, symmetry axis voting~\cite{xu2009}, eigenanalysis~\cite{lipman2010}, pseudo-polar Fourier transforms~\cite{bermanis2010}, etc.
For general partial symmetries, our algorithm can handle reflectional, rotational and translational problems in a unified framework. 
%%%ZQC Our approach has some similarities to the partial symmetry detection method of~\cite{berner2011}, but goes beyond it robustly producing more convincing results.
%%%ZQC I masked the sentence as it is some strong.
%%%RRM Too brief. Say how it is similar, how it differs, and why it differs
%%%RRM Say in what way the results (i) are more convincing (ii) are more robust.

The relationship between symmetry and segmentation can be viewed as a chicken-and-egg problem, as each provides clear guidance for the other.
Symmetry detection has already been used to provide more plausible segmented parts, in both global~\cite{podolak2006,podolak2007} and partial symmetry algorithms~\cite{xu2009,lipman2010}.
The output of most symmetry detection algorithms is expressed in terms of segmented object parts.
In this paper, we use segmentation to drive our symmetry detection algorithm.

