\RequirePackage{fix-cm}
%
%\documentclass{svjour3}                     % onecolumn (standard format)
%\documentclass[smallcondensed]{svjour3}     % onecolumn (ditto)
%\documentclass[smallextended]{svjour3}       % onecolumn (second format)
\documentclass[twocolumn]{svjour3}          % twocolumn
\smartqed  % flush right qed marks, e.g. at end of proof
\usepackage{graphicx}

\usepackage{amssymb,amsmath}
%\newcommand{definition}{[Definition]}
%\newtheorem{definition}{Definition}
\DeclareMathOperator{\Ker}{Ker}
\DeclareMathOperator{\Supp}{Supp}

%\ifpdf \usepackage[pdftex]{graphicx} \pdfcompresslevel=9
%\else \usepackage[dvips]{graphicx} \fi

%\PrintedOrElectronic

% prepare for electronic version of your document
%\usepackage{t1enc,dfadobe}

%\usepackage{egweblnk}
\usepackage{cite}
\usepackage{algorithm}
\usepackage[noend]{algorithmic}
\usepackage{subfigure}
\usepackage{epsfig}
\usepackage{color}
\definecolor{turquoise}{cmyk}{0.65,0,0.1,0.1}
\definecolor{purple}{rgb}{0.65,0,0.65}
\newcommand{\tofix}[1]{{\color{green}[FIX: #1]}}
\newcommand{\ZQC}[1]{{\color{red}[Zhiquan: #1]}}
\newcommand{\YC}[1]{{\color{blue}[Yin: #1]}}
\newcommand{\YKL}[1]{{\color{magenta}[Yukun: #1]}}
\newcommand{\RRM}[1]{{\color{purple}[Ralph: #1]}}

% For backwards compatibility to old LaTeX type font selection.
% Uncomment if your document adheres to LaTeX2e recommendations.
\let\rm=\rmfamily    \let\sf=\sffamily    \let\tt=\ttfamily
\let\it=\itshape     \let\sl=\slshape     \let\sc=\scshape
\let\bf=\bfseries

\journalname{CGI2012} % The correct name will be entered by the editor
\begin{document}

\title{Partial symmetry detection of 3d shapes}
%\subtitle{Insert your subtitle here, otherwise leave blank}
\subtitle{}
%\author{First Author \and Second Author}
\author{Paper ID: 18}
%\institute{F. Author \at first adress \and S. Author \at second address}
\institute{}
\date{ }% The correct dates will be entered by the editor

\maketitle

\begin{abstract}
In this paper, we propose a new algorithm for detecting symmetries in shapes.
Our algorithm computes generalized partial symmetries, i.e.,
subsets of a shape that reoccur multiple times within the model differing by combinations of translation, rotation and mirroring.
Our algorithm is based on segmentation and matching locally coherent meaningful parts on the object surfaces.
Working on relevant parts leads to a new algorithm that is able to detect partial symmetries more robust than the recent algorithms,
which are based on the correlated correspondences among graphs of invariant features.
We apply our algorithm to a number of 3D data sets, demonstrating high recognition rates for general partial symmetry.

\keywords{symmetry detection \and partial symmetry \and segmentation}
\end{abstract}

\section{Introduction}
\label{sec:intro}

\begin{figure}[t]
\centering
  \includegraphics[width=0.99\linewidth]{figures/Gargoyl.pdf}
  \caption{Two-scale symmetries of Gargoyl status from two viewpoints.
  Top: coarse-scale two pairs of mirroring symmetry from six segmented parts are represented, each pair is rendered by similar color.
  Below: fine-scale translation and rotation symmetry detections for the detailed rings.}
\label{fig:Gargoyl}
\end{figure}

Many objects, both man-made and natural, have some symmetries or self-similarities in them.
Finding the symmetry from triangle meshes is a way to augment meshes with some structures thus helps various applications such as remeshing~\cite{podolak2006}, mesh simplification~\cite{pauly2008}, segmentation~\cite{mitra2006,xu2009}, repairing~\cite{bokeloh2009,berner2011}, and reconstruction~\cite{zabrodsky1997}.
Symmetry detection and analysis is a fundamental technique in the computer graphics, computer vision and geometry processing.
Whilst a lot of attention has been received in recent years, it is still challenging to robustly identify symmetries in general input meshes without resorting to user assistance.

Symmetries may have different definitions. In this work, we focus on generalized partial symmetry~\cite{mitra2006,berner2011} where some subset of a shape that reoccurs multiple times within the model differing by combination of translation, rotation or mirroring. Intrinsic symmetries are also considered in which case further isometric deformation is allowed between copies. This flexible definition allows symmetry detection to be more useful.

In this paper, we propose an effective method for generalized partial symmetry detection.
The input mesh is first segmented into multiple meaningful pieces. The correspondence between every pair of pieces
is established using a matching and any pair with significantly high number of correspondences is recognized as a symmetric pair.

Compared to the recent graph matching algorithms~\cite{bokeloh2009,berner2011} based on the salient lines, our algorithm could produce more robust partial symmetry detection results.
As shown by Figure~\ref{fig:Gargoyl}, the two-scale symmetries of the Gargoyl status are correctly detected, which is the failure case of~\cite{berner2011}.
The top is the coarse-scale mirroring symmetry detection from six meaningful segmented parts, two pairs mirroring symmetries (represented by similar colors) have been found.
The below is the fine-scale symmetry detection on the left wing, both the translational and rotational symmetries have been found for the detailed rings.    

The rest of the paper is structured as follows. After the related algorithms are surveyed in Section 2, Section 3 describes our symmetry detection algorithm. 
Section 4 demonstrates some results and compares them with related works. Finally, section 5 makes conclusions and gives some discussions and future directions. 
\section{Related work}
\label{sec:related}

Symmetry detection and analysis have been thoroughly studied from theoretic, algorithmic, and applicative aspects during the last two decades.
There exists a significant number of papers proposed for this purpose, which could be mainly classified into \emph{global} and \emph{partial} types.
As we will aim to detect the intrinsic symmetry in 3D shapes in this paper, we concentrate on the most relevant 3D shape symmetry methods and recent advances in the intrinsic topic.
For the comprehensive review, please refer to papers~\cite{Xu2009,Berner2011}.

A common approach is to identify the global symmetry detection in Euclidean space to the clusters in the transformation space~\cite{mitra2006,podolak2006,pauly2008}. Other approaches~\cite{bokeloh2009,Berner2011} formulate the symmetry detection as partial graph matching, and compute the reliable line features from scanned data.
Extrinsic symmetry methods always employed symmetry of the Euclidean space for detecting symmetries~\cite{zabrodsky1997,mitra2006,podolak2006},
and patterns~\cite{pauly2008,bokeloh2009,yeh2009,Berner2011}.

Recent works have focused on analyzing partial and approximate symmetries, which is more difficult than the extrinsic, given the high dimensionality of the non-rigid transformation.
is more difficult than global extrinsic symmetries.
Similar to extrinsic symmetry, intrinsic symmetry could be characterized by global~\cite{ovsjanikov2008,bronstein2009,Ben-Chen2010,Bokeloh2010,Chertok2010,Lipman2010,Kim2010,Hooda2011,kim2011} and partial~\cite{Lasowski2009,Xu2009,Mitra2010,Raviv2010,Bronstein2011} cases. While from the transformation formulation viewpoint, intrinsic symmetry algorithms are striving for the compact representation using parameters, e.g., translation, rotation, scaling, and reflection. Except an alternative way of detecting global intrinsic symmetry is using the approximate killing vector field~\cite{Ben-Chen2010} from a continuous perspective, most algorithms considered discrete symmetries by sampling.
Ovsjanikov et al. perform global intrinsic symmetry detection using eigenanalysis of the surface Laplace-Beltrami operator~\cite{ovsjanikov2008}, then resorting to the heat kernel maps~\cite{Ovsjanikov2010}. By utilizing the Mobius transformation and conformal mapping, two robust symmetry detection algorithms were proposed by Kim and his coworkers~\cite{Kim2010,Kim2011}. Meanwhile, the Global MultiDimensional Scaling (GMDS) computation was thoroughly studied by Bronstein's researching team~\cite{bronstein2006,bronstein2009,Raviv2010,Bronstein2011} to detect the extrinsic and intrinsic symmetries, and mainly applied to the shape invariant retrieval applications. In addition, other researchers proposed variant symmetry methods by using different techniques, e.g., Markov random field model~\cite{Lasowski2009}, symmetry axis voting~\cite{Xu2009}, Eigen analysis~\cite{Chertok2010,Lipman2010}, Pseudo-polar Fourier Transform~\cite{Bermanis2010}, etc.

It is more challenging to detect the partial symmetry for 3D shapes. For the general partial symmetries, our algorithm would solve the reflectional, rotational and translational problems in one unified framework. Our approach is some similar to the symmetry detection using feature lines~\cite{Berner2011}, but going beyond it by stably handling more challenging input data. 
\section{Algorithm}
\label{sec:alg}
\section{Results}
\label{sec:result}

\begin{figure}[t]
\centering
  \includegraphics[width=0.99\linewidth]{figures/Rooster.pdf}
  \caption{Alignment of several Rooster scans from different viewpoints.
  Above: our final registered Rooster and the ground truth [Chuang et al. 2009]. Below: 8 partial scans, the dark lines indicating the pairwise matches.}
\label{fig:3DRigid}
\end{figure}

\section{Conclusions}
\label{sec:con}


\bibliographystyle{ieeetr}
\bibliography{parsym}
\end{document}


