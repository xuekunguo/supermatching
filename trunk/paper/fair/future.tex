\section{Conclusion}
\label{sec:conclusion}

This paper has presented the novel SuperMatching algorithm,
which tackles the classic computer graphics and computer vision problem of feature matching, independently of feature description. 
It is an efficient higher-order matching algorithm which uses a compact form of the higher-order supersymmetric affinity tensor to express relatedness of features.
Matching is performed using an efficient power iteration method, whose efficiency takes advantage of supersymmetry and avoids computing with zero elements.
We also give an efficient sampling strategy for choosing feature tuples to create the affinity tensor.
Experiments on both synthetic and real 2D and 3D data sets show that
SuperMatching has greater accuracy than competing methods, whilst having competitive performance.

%%===========================================================
%% The Appendices part is started with the command \appendix;
%% appendix sections are then done as normal sections
%\appendix
%\section{Why higher-order power iteration solving of supersymmetric tensor is efficient?}
%We explain the idea using a third-order affinity tensor as an example.

% The general matching algorithm finds correspondences considering triangle or higher-order polygons formed by feature points, going beyond traditional pointwise and pairwise approaches. It is formulated as a supersymmetric-tensor-based matching scheme, solved by an efficient higher-order power iteration method. Experiments with several applications demonstrante its accuracy.
