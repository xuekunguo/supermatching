\section{Related work}
\label{sec:related}

Symmetry detection and analysis have been thoroughly studied from theoretic, algorithmic, and applicative aspects during the last two decades.
There exist a significant number of papers proposed for this purpose, which can mainly be classified into \emph{global} and \emph{partial} types.
As we will aim to detect the partial symmetry in 3D shapes in this paper, we concentrate on the most relevant 3D shape symmetry methods and recent advances in the segmentation-symmetry aspect.
For the comprehensive review, please refer to papers~\cite{xu2009,berner2011}.

A common approach for symmetry detection is to identify the global symmetry in Euclidean space from the clusters in the transformed space~\cite{podolak2006,pauly2008}. 
Other approaches~\cite{Berner2008,bokeloh2009} formulate the symmetry detection as graph matching, and compute the reliable line features from scanned data.
Extrinsic symmetry methods always employed symmetry of the Euclidean space for detecting symmetries~\cite{zabrodsky1997,podolak2006,ovsjanikov2008,ovsjanikov2010,chertok2010,hooda2011},
and patterns~\cite{pauly2008,bokeloh2009,yeh2009}.
Ovsjanikov et al. perform global intrinsic symmetry detection using eigenanalysis of the surface Laplace-Beltrami operator~\cite{ovsjanikov2008}, 
then resorting to the heat kernel maps~\cite{ovsjanikov2010}.
By utilizing the Mobius transformation and conformal mapping, two robust symmetry detection algorithms were proposed by Kim and his coworkers~\cite{kim2010,kim2011}.
Most algorithms considered discrete symmetries by sampling. An exception is the work~\cite{ben-chen2010}
where global intrinsic symmetry is detected using the approximate killing vector field~\cite{ben-chen2010} from a continuous perspective.

Recent works have focused on analyzing partial and approximate symmetries, which is more difficult than the global, given the partial data.
Similar to global symmetry, partial symmetry detection can also be categorized as extrinsic~\cite{mitra2006,bronstein2009,berner2011} and intrinsic~\cite{lasowski2009,xu2009,mitra2010,raviv2010,bronstein2011} cases.
The Global MultiDimensional Scaling (GMDS) computation was thoroughly studied by Bronstein and coworkers~\cite{bronstein2006,bronstein2009,raviv2010,bronstein2011} to detect the extrinsic and intrinsic symmetries, and mainly applied to the shape invariant retrieval applications. In addition, other researchers proposed variant symmetry methods by using different techniques, e.g., Markov random field model~\cite{lasowski2009}, symmetry axis voting~\cite{xu2009}, Eigen analysis~\cite{lipman2010}, Pseudo-polar Fourier Transform~\cite{bermanis2010}, etc.
For the general partial symmetries, our algorithm would solve the reflectional, rotational and translational problems in a unified framework. Our approach is some similar to the partial symmetry detection~\cite{berner2011}, but going beyond it by robustly producing more reasonable results.

The relationship between symmetry and segmentation could be viewed as chicken-and-egg, as they are foundation and output for each other. 
The directive application of symmetry detection is the more plausible segmented parts, as demonstrated by the global~\cite{podolak2006,podolak2007} and partial algorithms~\cite{xu2009,lipman2010}.
At the same time, the output of symmetry detection is always segmented parts, proved by most previous algorithms.
In the paper, the fact that segmentation is always the foundation of symmetry detection is clearly claimed, and use it to develop new symmetry detection algorithm. 

