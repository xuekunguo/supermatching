\section{Related work}
\label{sec:related}

Previous approaches to feature matching can be classified into those which match single points to single points, those which match pairs of points to pairs of points, and so on.

Matching single points to single points is a linear assignment problem which only considers an affinity measure between two features, one from each set being matched.
Affinity measures used in  computer graphics and computer vision tasks are defined typically as the feature distance between feature vectors based on local information around each feature point,
e.g.\ SIFT~\cite{Lowe04}, spin images~\cite{Johnson99}, slippage features~\cite{Bokeloh08}, heat diffusion signatures~\cite{Sun09}, and BRISK~\cite{Leutenegger11}.
Point-to-point matching can give misleading results as wrong correspondences are readily established.

Matching pairs of points in one set to pairs of points in the other set leads to a quadratic assignment problem.
The usual approach is now to take into account both similarity of the point features \emph{and} the Euclidean distance between the points in a pair,
assuming the objects are related by a rigid transformation~\cite{Leordeanu05,Cour06}, or geodesic distance, assuming isometry~\cite{li08,Tevs09,Ovsjanikov10,Tevs11,SahilliogluY11,Windheuser11}.
Unfortunately, this quadratic assignment problem is NP-hard, and again, matches found are not always reliable.

Several higher-order approaches have also been proposed.
While they can significantly improve matching accuracy,
higher-order assignment  is even more computationally demanding, and various approximate methods have been developed.
\cite{Zass08} considered a probabilistic model of soft hypergraph matching.
They reduce the higher-order problem to a first-order one by marginalizing the higher-order tensor to a one dimensional probability vector.
\cite{Duchenne09} introduced a third-order tensor in place of an affinity matrix to represent affinities of feature triples,
and higher-order power iteration was used to achieve the final matching.
\cite{Chertok10} treated the tensor as a joint probability of assignments, marginalize the affinity tensor to a matrix,
and find optimal soft assignments by eigendecomposition of the matrix.
\cite{Aiping10} also built a multiple higher-order affinity tensor, and obtain a final matching by rank-one approximation of the tensor.
Higher-order assignment problems typically require large amounts of memory and computational resources. By reducing the number of elements needed to represent the affinity measures, the above approaches can match moderate numbers (many hundreds) of features. However, these 2D approaches do not take advantage of supersymmetry of the affinity tensor, SuperMatching does so, leading to an improvement in matching accuracy.
3D problems are even more challenging.

A related idea using higher order constraints in 3D registration, the 4-points congruent sets method (4PCS), was proposed by Aiger et al.~\cite{Aiger08}.
It is a fast alignment scheme for 3D point sets that uses widely separated points.
However, the need to find coplanar 4-tuples of points and the assumption of rigid transformation limit its applicability.
We solve both rigid and isometric shape matching problems with a single approach.
