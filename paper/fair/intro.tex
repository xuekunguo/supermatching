\section{Introduction}
\label{sec:introduction}

Building correspondences between two sets of features belonging to a pair of 2D and 3D shapes or images
is a fundamental problem in many computer graphics, geometry processing, and computer vision tasks.
It arises in applications such as
registration of 3D shapes~\cite{Gelfand05,Aiger08,li08,Zeng10,vanKaick11,Chang11},
shape retrieval from databases~\cite{Bronstein11},
shape matching~\cite{Berg05,Brown07,Lorenzo08,Tevs09,Ovsjanikov10,Tevs11,SahilliogluY11,Windheuser11},
shape reconstruction~\cite{Brown07,Pekelny08,Wand09,Chang11},
and automatic shape understanding~\cite{Lipman09,Sun10,Kim11} .

In principle and practise, determining correspondences is typically done using three steps~\cite{Johnson99,Lowe04,Sun09,Bokeloh08,Toler10,Leutenegger11}:
(i) computing high-quality descriptors which serve to distinguish points from one another,
(ii) choosing certain salient points with unusual feature descriptors, for matching,
and (iii) determining the most suitable matching between the two sets of points.
The former two problems have widely attracted considerable attention as their importance is easy perceivable.
However, even supposing ideal feature  descriptors and selectors  that capture the most important and distinctive information about the neighborhood of each salient point,
state-of-the-art algorithms  still find it challenging to determine the best matching~\cite{vanKaick11}.
The reasons are various:  real input data is noisy, the data may only be approximately in correspondence, 
and the problem is further complicated by the presence of (nearly) symmetric and congruent regions.
Feature matching algorithms need to be robust in the presence of such issues.
Various approaches have been devised with this in mind,
such as RANSAC-like algorithms~\cite{Tevs09,Tevs11} to minimize the effects of outliers,
generalized multidimensional scaling~\cite{Bronstein11} and
heat kernel maps~\cite{Ovsjanikov10} which consider the manifold in which the points are embedded, and also M{\"o}bius transformations~\cite{Lipman09,Kim11}.
However, these previous algorithms still do not treat the matching step as an independent problem, even if in these cases matching is not tightly coupled with feature description and selection.

In the paper, we focus on the feature matching problem, as a problem in its own right.
Matching may be done pointwise, or using tuples of points.
We may match single points to single points (point-single),
point pairs separated by a fixed distance to other point pairs (line segment-double),
triples of points forming a triangle to other triples of points (triangle-triple), and so on.
%quadruples of points forming a quadrangle to other quadruples of points (quadrangle-quadruple), and so on.

As pointed by~\cite{Conte04}, when single features are matched,
we must solve a linear assignment problem, but if multiple features are matched at once,
a quadratic or higher-order assignment problem results.
Linear assignment matches single features in one set with single features in the other set.
Matching two feature sets by considering similarities of \emph{single} features from each set can easily fail in the presence of ambiguities such as repeated elements,
or similar local appearance.
Quadratic and higher-order assignment matches groups of features in one set simultaneously with groups from the other set,
and requires a greater consistency between the information being matched, making it more reliable.
As well as the features themselves, other constraints such as consistency of the distances between the features being matched are also enforced,
greatly improving the matching accuracy.
In general, we formulate these constraints by modelling the affinity relating the two point sets.

As a particular example of \emph{quadratic} assignment, Leordanu and Hebert~\cite{Leordeanu05} consider pairs of feature descriptors,
and use distances between pairs of features from each set to reduce the number of incorrect correspondences.
Such pairwise distance constraints are particularly helpful in cases when the features themselves have low discriminative ability.
The idea has been widely adopted in 3D shape matching algorithms~\cite{Tevs09,Ovsjanikov10,Tevs11,Kim11,SahilliogluY11,Windheuser11}.

Higher-order assignment further generalizes the assignment problem to include yet more complex constraints between features.
For example, third-order potential functions, proposed in~\cite{Duchenne09,Zeng10,Chertok10},
quantify the affinity between two point triples by measuring the similarity of the angles of the triangles generated by such triples.
However, this angular similarity value only considers the total difference in corresponding angles, and does not change according to ordering of elements in the tuple.
By changing the affinity tensor to a \emph{supersymmetric} tensor~\cite{Kofidis02}, this limitation is overcome by our algorithm.

To summarize, our \emph{SuperMatching} algorithm formulates the higher-order matching problem using a supersymmetric affinity tensor. It can accurately match a moderate number
of features using triples or greater tuples of features.
The contributions of this paper include:
\begin{itemize}
\item We show how to define a compact higher-order supersymmetric affinity tensor to express geometric consistency constraints between tuples of features.

\item Relying on the supersymmetry of the affinity tensor, we give a higher-order power iteration method which efficiently solves the matching problem.

\item The affinity tensor is estimated by using a
new efficient sampling strategy for feature tuples which avoids sampling repetitive items,  both reducing the number of feature tuples to be sampled and improving the matching accuracy.

\end{itemize}

Our experiments given later for both synthetic and real captured data sets show that SuperMatching is accurate and robust,
while having a competitive computational cost compared to previous algorithms.
Importantly, the matching approach is general as it is independent of choice of 2D or 3D feature descriptors and feature point selection method.
