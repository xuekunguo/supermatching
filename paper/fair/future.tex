\section{Conclusion}
\label{sec:conclusion}

This paper has given a novel matching algorithm named SuperMatching,
which tackles the classic Computer Graphics and Computer Vision problem of matching various features for cases without any assumptions.
SuperMatching is an efficient higher-order matching algorithm based on the supersymmetric affinity tensor.
Our main contributions are as follows.
First, we use a  higher-order supersymmetric affinity tensor with a compact form to express higher-order consistency constraints of features.
Secondly, we derive an efficient higher-order power iteration method, which makes significant efficiency by taking advantage of supersymmetry.
Finally, we also give an efficient sampling strategy for choosing feature tuples to create the affinity tensor.
The experiments on both synthetic and real 2D/3D data sets show that SuperMatching is one general feature matching algorithm with accurate performance.

This paper has presented a novel matching algorithm, SuperMatching,
which tackles the classic computer graphics and computer vision problem of matching various features between images and surfaces.
It is an efficient higher-order matching algorithm which takes advantage of the supersymmetry of the affinity tensor. 
Supersymmetry is used to both efficiently sample the affinity tensor and to store it compactly. 
An efficient higher-order power iteration method taking advantage of supersymmetry is used to perform the matching.
It is independent of feature vector definition,
feature point selection method, and nature of the geometric transformations and constraints linking feature points.
Our experiments on both synthetic and real 2D/3D data sets show that SuperMatching is a general feature matching algorithm which is accurate and robust, whilst having competitive performance.


%%===========================================================
%% The Appendices part is started with the command \appendix;
%% appendix sections are then done as normal sections
%\appendix
%\section{Why higher-order power iteration solving of supersymmetric tensor is efficient?}
%We explain the idea using a third-order affinity tensor as an example.

% The general matching algorithm finds correspondences considering triangle or higher-order polygons formed by feature points, going beyond traditional pointwise and pairwise approaches. It is formulated as a supersymmetric-tensor-based matching scheme, solved by an efficient higher-order power iteration method. Experiments with several applications demonstrante its accuracy.
