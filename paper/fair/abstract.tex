Feature matching is a challenging problem lying at the heart of numerous computer graphics and computer vision applications.
We present here the \emph{SuperMatching} algorithm for finding correspondences between two sets of features.
It does so by considering triangles or higher-order tuples of points, going beyond the pointwise and pairwise approaches typically used.
SuperMatching is formulated using a supersymmetric tensor representing an affinity metric which takes into account geometric constraints between features:
feature matching is cast as a higher-order graph matching problem.
SuperMatching takes advantage of supersymmetry to devise an
efficient sampling strategy to estimate the affinity tensor, as well as to store the tensor compactly.
Matching is performed by computing a rank-one approximation of the tensor directly using a higher-order power iteration solution.
\cz{Matching is performed by an efficient higher-order power iteration solution, deduced from the supersymmtric affinity tensor.}
%%%YKL In addition to describe what the whole system is, I think we should emphasise the contributions in the abstract, like supersymmetry.
%%%YKL If you just say rank-one approximation, high-order power iteration, etc. they are not unique from previous methods.
%%%ZQC \cz
Experiments on both synthetic and real captured data show that
SuperMatching provides accurate feature matching for a wide range of 2D and 3D features,
giving more reliable results than other state-of-the-art approaches, with competitive computational cost. 