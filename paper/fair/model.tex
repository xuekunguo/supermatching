%%==========================================================================
\section{Overview}
\label{sec:overview}

Assume we are given two sets of feature points $P$ and $Q$, with $N_1$  and $N_2$ points respectively.
The matching between these two feature sets can be represented by an \emph{assignment variable} $\mathbf{X}$.
$\mathbf{X}$ is a matrix, whose size is $N_1N_2$ and elements are 0 or 1.
$X(p,q)=1$ when point $p \in P$ matches point $q \in Q$ and $X(p,q)=0$ otherwise.
$\mathbf{X}$ can be row-wise vectorized to give an assignment vector $\boldsymbol{x} \in \{0,1\}^N$ where $N={N_1N_2}$.

Solving the higher-order matching problem is equivalent to finding the optimal assignment vector ${\boldsymbol{x}}^*=<x_{i_1},\cdots,x_{i_N}>
 \in \{0,1\}^{N}$, satisfying
\begin{equation}
\label{equ:assigment}
  {\boldsymbol{x}}^* = \argmax_{\boldsymbol{x}}  \sum_{i_1,\cdots,i_N} \mathcal{T}_r(i_1,\cdots,i_N) x_{i_1}  \cdots\; x_{i_N}.
\end{equation}

Here, $i_n$ stands for an assignment $(p_n,q_n)$, $(p_n\in P, q_n\in Q)$, and the
product $x_{i_1} \cdots\;x_{i_N}$ means the tuple of features $(p_1,\cdots,p_{N_1})$ from $P$ is matched to the tuple of features $(q_1,\cdots,q_{N_2})$ from $Q$.
$\mathcal{T}_r(i_1,\cdots,i_N)$ defines the affinity of the set of assignments $\{i_n\}_{n=1}^N$;
it is the affinity measure between the ordered feature tuples $(p_1,\cdots,p_{N_1})$  and $(q_1,\cdots,q_{N_2})$.
In the paper, \emph{the affinity measures would be defined based on the geometric constraints between pairs of feature tuples},
this is the basis of the SuperMatching algorithm.

In the rest of the paper, we consider the one-to-many correspondence problem.
We assume that each point in $P$ is matched to exactly one point in $Q$, but that the reverse is not necessarily true.
If \emph{do} we want to treat both datasets in the same way,
we can first match $P$ to $Q$, then match $Q$ to $P$, and then combine the matching results by taking their union or intersection.
Uniqueness of matches for $P$ means that the assignment variable matrix $\mathbf{X}$ satisfies $\sum\nolimits_p X(p,q)=1$.

From Equ.(\ref{equ:assigment}) we can see that there are four issues to be considered when using higher-order matching algorithms. How should we:
\begin{itemize}
\item organize and express the affinity measures $\mathcal{T}_N$ in a storage efficient manner? (see Section~\ref{subsec:supersymtensor})
\item approximately solve the optimal higher-order assignment problem efficiently? (see Section~\ref{subsec:oursymmhopm})
\item define the affinity measure between two feature tuples? (see Section~\ref{subsec:potentials})
\item determine an appropriate sampling strategy to estimate the affinity tensor in a way which will give good matching accuracy? (see Section~\ref{subsec:sampling})
\end{itemize}



