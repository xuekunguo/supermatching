%%==========================================================================
\section{Overview}
\label{sec:overview}

Assume we are given two sets of feature points $P_1$ and $P_2$, with $N_1$  and $N_2$ points respectively.
The matching between these two feature sets can be represented by an \emph{assignment matrix} $X$ of size $N_1 \times N_2$ whose elements are 0 or 1.
$X(i,j)=1$ when $i \in P_1$ matches $j \in P_2$ and $X(i,j)=0$ otherwise. $X$ can be row-wise vectorized to give an assignment vector $\boldsymbol{x} \in \{0,1\}^{N_1N_2}$.

Solving the higher-order matching problem is equivalent to find the optimal assignment vector ${\boldsymbol{x}}^*=<x_{i_1},\cdots,x_{i_N}> \in \{0,1\}^{N_1N_2}$, satisfying
\begin{equation}
\label{equ:assigment}
  {\boldsymbol{x}}^* = \argmax_{\boldsymbol{x}}  \sum_{i_1,\cdots,i_N} \mathcal{T}_N(i_1,\cdots,i_N) x_{i_1}  \cdots\; x_{i_N}.
\end{equation}
Here, $i_n$ stands for an assignment $(i^{'}_n,j^{'}_n)$, $(i^{'}_n\in P_1,j^{'}_n\in P_2)$, and the
product $x_{i_1}  \cdots\;x_{i_N}$ is 1 only if all assignments $\{i_n\}_{n=1}^N$ are equal to 1, which means the tuple of features $(i^{'}_1,\cdots,i^{'}_N)$ from $P_1$ is matched correspondingly to the tuple of features $(j^{'}_1,\cdots,j^{'}_N)$ from $P_2$.
$\mathcal{T}_N(i_1,\cdots,i_N)$ defines the affinity of the set of assignments $\{i_n\}_{n=1}^N$;
it also can be seen as the affinity measure between the ordered feature tuples $(i^{'}_1,\cdots,i^{'}_N)$  and $(j^{'}_1,\cdots,j^{'}_N)$.

In the paper, we consider the one-to-many correspondence problem.
We assume that a point in $P_1$ is matched to exactly one point in $P_2$, but that the reverse is not necessarily true.
If \emph{do} we want to treat both datasets in the same way,
we can first match $P_1$ to $P_2$, then match $P_2$ to $P_1$, and then combine the matching results by taking their union or intersection.
Uniqueness of matches for $P_1$ means that the assignment matrix $X$ satisfies $\sum\nolimits_i X(i,j)=1$.

From Equ.(\ref{equ:assigment}) we can see that there are four issues to be considered when using higher-order matching algorithms. How should we:
\begin{itemize}
\item organize and express the affinity measures $\mathcal{T}_N$? (see Section~\ref{subsec:supersymtensor})
\item approximately solve the optimal higher-order assignment problem efficiently? (see Section~\ref{subsec:oursymmhopm})
\item define the affinity measure between two feature tuples, or equivalently, the higher-order potential function $\phi_N$? (see Section~\ref{subsec:potentials})
\item determine an appropriate sampling strategy that physically build the affinity tensor and influence good matching accuracy? (see Section~\ref{subsec:sampling})
\end{itemize}



