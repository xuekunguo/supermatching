%%==========================================================================
\section{Overview}
\label{sec:overview}

A tensor generalizes vectors and matrices to higher dimensions: a vector is a tensor of order one,
and a matrix is a tensor of order two. A higher-order tensor can be expressed as a multi-dimensional array~\cite{Kolda08}.
More formally, an $N^{th}$-order tensor is an element of the tensor product of $N$ vector spaces, each with its own coordinate system.

Assume we are given two sets of feature points $P_1$ and $P_2$, with $N_1$ and $N_2$ points respectively.
$i_n = (f^1_{i_n}, f^2_{i_n})$ is a pair of points from $P_1$ and $P_2$, respectively.
Matching between these two feature sets can be represented by an \emph{assignment variable} $\boldsymbol{x}$ which is a vector $\in \{0,1\}^{N_1 N_2}$, with each element representing
whether a pair $i_n(f^1_{i_n}, f^2_{i_n})$ is selected in the matching (if $x_{i_n} = 1$) or not (if $x_{i_n} = 0$).
From the $N^{th}$-order tensor viewpoint,
the matching problem is equivalent to finding the optimal assignment tensor ${\boldsymbol{x}}^*
 \in \{0,1\}^{N_1 N_2}$, satisfying~\cite{Kolda08}
\begin{equation}
\label{equ:assigment}
  {\boldsymbol{x}}^* = \argmax_{\boldsymbol{x}}  \sum_{i_1,\cdots,i_N} \mathcal{T}_N(i_1,\cdots,i_N) x_{i_1}  \cdots\; x_{i_N}.
\end{equation}
Here, $i_n \in \{i_1,\cdots ,i_N\}$ stands for an assignment in the $n^{th}$ dimension of the $N$ vector spaces.
Let all feature tuples for $P_1$ and $P_2$ be $F_1$ and $F_2$, then $\forall (f_{i_1}^1, \cdots, f_{i_N}^1)\in F_1$,
there is a matching to corresponding feature tuples in $F_2$.
For example, given a third-order tensor, $i_n \in \{1,2,3\}$,
each index could be expressed as $i_1=(f_{i_1}^1,f_{i_1}^2), i_2=(f_{i_2}^1,f_{i_2}^2), i_3=(f_{i_3}^1,f_{i_3}^2)$: pairs of potentially matched points.
The product $x_{i_1} \cdots\;x_{i_N}$ will be equal to $1$ if the points $(f_{i_1}^1, \cdots, f_{i_N}^1)$ are matched to the points $(f_{i_1}^2, \cdots, f_{i_N}^2)$,
and otherwise 0.
$\mathcal{T}_N(i_1,\cdots,i_N)$ is the affinity of the set of assignments $\{i_n\}_{n=1}^N$,
which is high if the features in tuple $(f_{i_1}^1, \cdots, f_{i_N}^1)$  have similar descriptor values to the features in the tuple $(f_{i_1}^2, \cdots, f_{i_N}^2)$,
and their distance constraints are similar.
Note that the size of $\mathcal{T}_N(i_1,\cdots,i_N)$ is ${(N_1N_2)}^N$.
In this paper, the affinity measures expressing similarity of feature tuples are compactly represented and efficiently computed by using the supersymmetric tensor.

In the rest of the paper, we consider the one-to-many correspondence problem.
We assume that each point in $P_1$ is matched to exactly one point in $P_2$, but that the reverse is not necessarily true.
If we \emph{do} want to treat both datasets in the same way,
we can first match $P_1$ to $P_2$, then match $P_2$ to $P_1$, and then combine the matching results by taking their union or intersection.

From Equ.(\ref{equ:assigment}) we can see that there are four issues to be considered when using higher-order matching algorithms. How should we:
\begin{itemize}
\item organize and express the affinity measures $\mathcal{T}_N$ in a supersymmetric manner? (see Section~\ref{subsec:supersymtensor})
\item approximately solve the optimal higher-order assignment problem efficiently? (see Section~\ref{subsec:oursymmhopm})
\item determine an appropriate sampling strategy to estimate the affinity tensor in a way which will give good matching accuracy (it is too large to compute fully)? (see Section~\ref{subsec:sampling})
\item define a suitable affinity measure between two feature tuples? (see Section~\ref{subsec:potentials})\end{itemize}



